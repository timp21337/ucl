\documentclass [12pt, a4paper, twoside, titlepage] {article}

\usepackage{booktabs}% http://ctan.org/pkg/booktabs
\usepackage{graphicx}% http://ctan.org/pkg/graphicx
\usepackage{tabularx}
\usepackage{varwidth}% http://ctan.org/pkg/varwidth
\usepackage[T1]{fontenc}
\usepackage{lmodern}
\usepackage[top=2.5cm, bottom=2.5cm, left=2.5cm, right=2.5cm, footskip=2cm]{geometry}
\usepackage{lastpage}
\usepackage{fullpage}
\usepackage{moreverb}
\usepackage{alltt}
\usepackage{boxedminipage}
\usepackage[usenames,dvipsnames]{color} % Required for specifying custom colors and referring to colors by name
\usepackage{fancyhdr}
\usepackage{lastpage}
\usepackage{pdfpages}
\usepackage{hyperref}
\usepackage{listings}
\usepackage{placeins}
\usepackage{sectsty}
\usepackage{pdfpages}
\newcommand{\turn}[3][10em]{% \turn[<width>]{<angle>}{<stuff>}
  \rlap{\rotatebox{#2}{\begin{varwidth}[t]{#1}\bfseries#3\end{varwidth}}}%
}




\pagestyle{fancy}
\lfoot{\sffamily\documentName}
\cfoot{\sffamily\thepage\ of \pageref*{LastPage} }
\rfoot{\sffamily\documentVersion \date{\today}}

\newcommand{\context}{\sffamily \textcolor{blue}{\Large Context-Computing}}
\chead{\context}
\lhead{}
\rhead{}

\headheight 20pt
\headsep 10pt

\raggedbottom
\hypersetup{colorlinks=true, linkcolor=blue}


%----------------------------------------------------------------------------------------

\usepackage{listings} % Required for inserting code snippets

\definecolor{DarkGreen}{rgb}{0.0,0.4,0.0} % Comment color
\definecolor{highlight}{RGB}{255,251,204} % Code highlight color

\lstdefinestyle{Style1}{ % Define a style for your code snippet, multiple definitions can be made if, for example, you wish to insert multiple code snippets using different programming languages into one document
language=Perl, % Detects keywords, comments, strings, functions, etc for the language specified
backgroundcolor=\color{highlight}, % Set the background color for the snippet - useful for highlighting
basicstyle=\footnotesize\ttfamily, % The default font size and style of the code
breakatwhitespace=false, % If true, only allows line breaks at white space
breaklines=true, % Automatic line breaking (prevents code from protruding outside the box)
captionpos=b, % Sets the caption position: b for bottom; t for top
commentstyle=\usefont{T1}{pcr}{m}{sl}\color{DarkGreen}, % Style of comments within the code - dark green courier font
deletekeywords={}, % If you want to delete any keywords from the current language separate them by commas
%escapeinside={\%}, % This allows you to escape to LaTeX using the character in the bracket
firstnumber=1, % Line numbers begin at line 1
frame=single, % Frame around the code box, value can be: none, leftline, topline, bottomline, lines, single, shadowbox
frameround=tttt, % Rounds the corners of the frame for the top left, top right, bottom left and bottom right positions
keywordstyle=\color{Blue}\bf, % Functions are bold and blue
morekeywords={}, % Add any functions no included by default here separated by commas
numbers=left, % Location of line numbers, can take the values of: none, left, right
numbersep=10pt, % Distance of line numbers from the code box
numberstyle=\tiny\color{Gray}, % Style used for line numbers
rulecolor=\color{black}, % Frame border color
showstringspaces=false, % Don't put marks in string spaces
showtabs=false, % Display tabs in the code as lines
stepnumber=5, % The step distance between line numbers, i.e. how often will lines be numbered
stringstyle=\color{Purple}, % Strings are purple
tabsize=2, % Number of spaces per tab in the code
}

% Create a command to cleanly insert a snippet with the style above anywhere in the document
% The first argument is the script location/filename and the second is a caption for the listing
%\newcommand{\insertcode}[3]{\begin{itemize}\item[]\lstlisting[caption=#2,label=#1,style=Style1]{#3}\end{itemize}} 

%----------------------------------------------------------------------------------------



% To enable reuse the following commands should be used in the body of the document, 
% rather than the strings they represent
\newcommand*{\defined}{\textbf}
\newcommand{\documentVersion}{V 1.0 }
\newcommand{\projectShortProperName}{TheBookSeekers.com}
\newcommand{\projectShortName}{TheBookSeekers}
\newcommand{\projectTitle}{Initial Childrens Book Recommendation Site}
\newcommand{\documentTypeTitle}{Proposal}
\newcommand{\documentName}{\documentTypeTitle \ - \projectShortProperName }
\newcommand{\authorName}{Tim Pizey}
\newcommand{\authorEmail}{Tim.Pizey@gmail.com}



\begin{document}
\sffamily
\allsectionsfont{\sffamily}


\title{
{\Huge \documentTypeTitle}
\\
\projectShortProperName 
\\ 
  {\small \projectTitle }
}

\author{
\context
\\\authorName
\\\tt{\authorEmail}
}
\date{} % no date
\maketitle
\clearpage
\newpage

\begingroup
\hypersetup{linkcolor=blue}
\tableofcontents
\endgroup
\newpage


\section{Introduction}
The project is intended to launch 1st October 2013, a very short time scale. The project is 
constrained in manpower to one developer. This implies that the ambition of the project must 
be limited to what is achievable in the available time.
The design of the system is not finalised, and is expected to be arrived at using Agile methodologies. This requires a time and materials approach to costs. 

\section{Technology choices}
The project will be implemented in Django, a python framework for delivering database backed websites. 

The project will be deployed to Heroku. 

The project will initially be house on an open github repository. 

The project will use a Jenkins installation \cite{jenkins.paneris.net}, supplied by Context-Computing in the first instance, 
for continuous integration and deployment.

\section{Initiation - take on of database schema}
This phase will deliver a Django CMS of the database structure, enabling administrators to edit the database. 

The database will be deployed to Heroku. 

an initial database schema has been included in the brief, this will be used as a guide and is expected to be amplified. 

\section{Deployment of wireframe}
A wire frame enabling 
\begin{itemize}
\item{Login}
\item{Registration}
\item{Display books alphabetically, paged}
\item{Search for a book}
\item{List collections}
\item{Display a collection}
\item{Collection Creation}
\item{Addition of a book to a collection}
\item{Create a tag}
\item{Associate a tag with a book}
\end{itemize}

Additional functionality will emerge, representing a risk to the projects timescales.

\section{Integration with Design}

This stage is dependent upon the delivery of the site design graphics. 

The amount of work in this stage is dependent upon the complexity of the page components envisaged and so represents a risk to the cost of the project.

\section{Integration with Amazon Data }

An initial load of the book table from Amazon. 

Nightly job to pull new book data.

This task will need investigation and hence is a project risk.

\section{Respond to Feedback}
It is anticipated that there will be a an Agile relationship with the client, CrowdSource Ltd, and so 
there is a budget fro responding to client feedback.

The extent of rework is a project risk.


\section{Exclusions}
This proposal does not include scaling up to potential traffic volumes. 
Whilst the technology stack does scale there is no provision for load balancing, database replication, database backup, site tuning for speed.


There is no provision for creation of tag clouds. 

There is no provision for integration with Google Books or Facebook.

There provision of a separate mobile site and a mobile application is not planned for phase one. 

\section{Costs}
The project will cost \pounds26,000, payable in four instalments of  \pounds5,000 at each month end and one of \pounds6,000 upon handover. 

\newpage

\begin{thebibliography}{scribe}

\bibitem[Anobii]{Anobii}
\url{http://en.wikipedia.org/wiki/Anobii}


\bibitem[BookArmy]{BookArmy}
 (now defunct)
\url{http://en.wikipedia.org/wiki/BookArmy}

\bibitem[Bibliomania]{Bibliomania}
 (now defunct)
\url{http://www.bibliomania.com}


\bibitem[Goodreads]{Goodreads}
\url{http://en.wikipedia.org/wiki/Goodreads}

\bibitem[Google Books]{GoogleBooks}
\url{http://en.wikipedia.org/wiki/Google_Books}

\bibitem[jenkins.paneris.net]{jenkins.paneris.net}
\url{http://jenkins.paneris.net}

\bibitem[LibraryThing]{Librarything}
\url{http://en.wikipedia.org/wiki/LibraryThing}


\bibitem[Shelfari]{shelfari}
\url{http://en.wikipedia.org/wiki/Shelfari}


\bibitem[weRead]{weRead}
\url{http://en.wikipedia.org/wiki/weRead}


\end{thebibliography}
\end{document}
